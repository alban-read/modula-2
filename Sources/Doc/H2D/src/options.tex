\chapter{Options Reference}
\label{options}

This chapter contains brief descriptions of all options that may be
specified in either
\ifonline \ref[configuration file]{config:cfg}.
\else     configuration file (See \ref{config:cfg}).
\fi
or \ProjectFile{}.

The general option syntax is:

\verb'    Option = "-" name ("+" | "-" | "=" (string | integer))'

Option names are case insensitive. Case is preserved when string
option values are stored or emitted, but is ignored when they are
compared.

{\bf Examples}

\begin{verbatim}
    -mergeall+
    -GENWIDTH=78
    -BackEnd=Common
\end{verbatim}

\section{File extensions and prefixes}
\label{DEFEXT}\label{DIREXT}\label{HEADEXT}
\label{MODEXT}\label{PRJEXT}\label{TREEEXT}
\label{DEFPFX}\label{MACPFX}
% !!! index

This group of options sets file name extensions and prefixes for H2D
input and output files.

\begin{center}
\begin{tabular}{lll}
\bf Option  & \bf Sets extension for...                  & \bf Default \\
\hline
\bf DEFEXT  & definition modules                         & \tt def        \\
\bf DIREXT  & \See{directive files}{}{rules:nonstandard} & \tt dir        \\
\bf HEADEXT & header files                               & \tt h          \\
\bf MODEXT  & implementation modules                     & \tt mod        \\
\bf PRJEXT  & \See{project files}{Chapter }{project}     & \tt h2d        \\
\bf TREEEXT & include tree files (see the \OERef{GENTREE} option) & \tt tre  \\       % ???
 & & \\
\bf Option  & \bf Sets prefix for...         & \bf Default \\
\hline
\bf DEFPFX  & \See{definition modules}{}{using:fit:c}           & none        \\
\bf MACPFX  & \See{macro prototype modules}{}{using:fit:native} & \_          \\
\end{tabular}
\end{center}

\section{Translation options}

\ifonline \else

\begin{center}
\begin{tabular}{llc}
\bf Option           & \bf Meaning                     & \bf Default \\
\hline
\OERef{BACKEND}      & target compiler back-end        & \tt Common  \\
\OERef{CHANGEDEF}    & enable retranslation            & \tt OFF     \\
\OERef{COMMENTPOS}   & preserved comments position     & undefined   \\
\OERef{CPPCOMMENTS}  & recognize C++ comments          & \tt ON      \\
\OERef{CSTDLIB}      & set CSTDLIB value               & \tt OFF     \\
\OERef{GENDIRS}      & extract non-stsndard directives & \tt OFF     \\
\OERef{GENENUM}      & enum translation mode           & \tt Const   \\
\OERef{GENLONGNAMES} & keep directory names            & \tt OFF     \\
\OERef{GENMACRO}     & produce macro prototype modules & \tt OFF     \\
\OERef{GENROVARS}    & translate constants to r/o vars & \tt OFF     \\
\OERef{GENSEP}       & separate merged headers         & \tt OFF     \\
\OERef{GENTREE}      & generate inclusion tree         & \tt OFF     \\
\OERef{GENWIDTH}     & limit output line length        & unlimited   \\
\OERef{MERGEALL}     & merge all headers               & \tt OFF     \\
\OERef{PROGRESS}     & enable progress indicator       & \tt OFF     \\
\end{tabular}
\end{center}
\fi

For each option, a set of possible values is given in parenthesis:

\begin{center}
\begin{tabular}{ll}
string  & an arbitrary sequence of characters \\
numeric & an unsigned integer number \\
boolean & ON or OFF
\end{tabular}
\end{center}

\ifonline \else
\begin{description}
\fi

\OptHead{BACKEND}{target compiler back-end} (Native/C/Common)

Enables generation of definition modules suitable for
either native-code compiler, translator to C, or both.
It also affects generation of additional modules
(See \ref{using:fit}).

Default: {\tt Common} - both native and translator.

\OptHead{CHANGEDEF}{enable retranslation} (boolean)

If this option is set OFF, H2D does not translate a header file if a
definition module corresponding to it already exists. Otherwise, H2D
produces a new definition module which may overwrite the old one.

See also \ref{config:red}.

Default: OFF - do not process already translated headers

\OptHead{COMMENTPOS}{preserved comments position} (numeric)

Sets starting position of comments preserved by H2D in output files.
Has effect only on comments which are placed next to declarations.

Default: delimit comments with a single blank.

\OptHead{CPPCOMMENTS}{recognize C++ comments} (boolean)

If this option is set ON, H2D recognizes C++-style
comments (started with {\tt '//'}) in header files.

Default: ON - recognize C++ comments.

\OptHead{CSTDLIB}{set CSTDLIB value} (boolean)

Sets value of the {\bf CSTDLIB} XDS option in output definition modules
corresponding to top-level header files (listed on the command line or
in a \ProjectFile{}). See \ref{using:fit:c} for more information.

Default: OFF - set {\bf CSTDLIB} off in definition modules.

\OptHead{GENDIRS}{extract non-standard directives} (boolean)

If this option is set ON, a file containing \See{non-standard preprocessor
directives}{}{rules:nonstandard} is produced for each header file.

Deafult: OFF - do not extract non-standard directives

\OptHead{GENENUM}{enum translation mode} (Const/Enum/Mixed)

Defines whether C enumerations should be unconditionally translated
to integer constants (\verb'Const') or Modula-2 enumeration types
(\verb'Enum'). If set to \verb'Mixed', only enumerations with default
(or explicitly specified, but matching default) constant values are
translated to enumeration types; all other are translated to constants.
See also \ref{rules:types:enum}.

Default: \verb'Const' - always translate to constants

\OptHead{GENLONGNAMES}{keep directory names} (boolean)

This option controls translation of the {\tt \#include} directive in cases
when a specified file name contains directories. If this option is set OFF, H2D
strips directory names. Otherwise directory names are kept and separators
are replaced with underscore characters:

\verb'#include <sys\errno.h>          IMPORT sys_errno;'

See also \ref{project:name}.

Default: OFF - strip directory names

\OptHead{GENMACRO}{produce macro prototype modules} (boolean)

Setting this option ON forces H2D to produce prototype modules for
function-like C macros encountered in header files
({\em prototype} means that implementation modules contain procedures
with empty bodies; the actual code has to be written by hand).
These modules have not to be imported, but to be added to the link list.
See also \ref{using:fit:native}.

This option is ignored if the target is XDS-C (the \OERef{BACKEND} option
is set to \verb'C').

Default: OFF - do not generate macro modules.

\OptHead{GENROVARS}{translate constants to read-only variables} (boolean)

Being set ON, enables translation of \verb'#define'd constants
to read-only variables. Have no effect in native back-end sections.
See \ref{using:modrules:constants} for more information.

Default: OFF - do not translate constants to variables.

\OptHead{GENSEP}{insert merged headers separators} (boolean)

Setting this option ON forces H2D to insert a comment containing name
of a merged header before declarations from that header.

Default: OFF - do not insert separators

\OptHead{GENTREE}{generate inclusion tree} (boolean)

If this option is set ON, H2D produces a text file containing
a tree of \verb'#include' directives for each header file specified
on the command line or in a project file using the \ProjDir{module}.
A name of a tree file is a name of a corresponding header file
with extension defined by the \OERef{TREEEXT} option.

Default: OFF - do not produce tree files

\OptHead{GENWIDTH}{limit output line length} (numeric)

Sets maximum length of a string in the output file.

Default: do not limit string length.

\OptHead{MERGEALL}{merge all headers} (boolean)

If this option is set ON, H2D merges all header files included into the
translated header by means of the {\tt \#include} directive.
If this option is set OFF, H2D generates separate definition module
for each header which is not specified in the {\tt \#merge} directive.

See also \ref{rules:pp:include} and \ref{using:merge}.

Default: OFF - do not merge headers which are not specified
               in the {\tt \#merge} directive.

\OptHead{PROGRESS}{enable progress indicator} (boolean)

Setting this option ON enables progress indicator.

Default: OFF - show no progress indicator

\ifonline \else
\end{description}
\fi

\section{Base types definition}

The base types definition and mapping options, {\bf CTYPE} and {\bf M2TYPE},
have a special syntax. Either of them may be specified more than once
in a project or configuration file, provided that each time a new type is
defined. See \ref{using:modrules:mapping} for more information on usage
of these directives.

By default, H2D is configured to support XDS compilers.

\ifonline \else
\begin{description}
\fi

\OptHead{CTYPE}{define a C base type} (special)

This option defines size and default mapping of a C base type.

\begin{verbatim}
CTypeOption = '-' 'CTYPE' '=' Type '=' size ',' qualident
Type = 'signed char'     | 'unsigned char'      |
       'short int'       | 'unsigned short int' |
       'signed int'      | 'unsigned int'       |
       'signed long int' | 'unsigned long int'  |
       'float'           | 'double'             |
       'long float'      | 'long double'
\end{verbatim}

\verb'size' is the size of \verb'Type' in bytes, and \verb'qualident' is
a Modula-2 type to which \verb'Type' should be translated by default:

\verb'-CTYPE = signed char = 1, CHAR'

See \ref{using:modrules:mapping} for more information.

\OptHead{M2TYPE}{define a \mt{} type} (special)

This option defines a \mt{} type.

\begin{verbatim}
M2TypeOption = '-' 'M2TYPE' '=' qualident '=' size ',' Attr
Attr = 'BOOL'   | 'CHAR' |
       'REAL'   | 'SET'  |
       'SIGNED' | 'UNSIGNED'
\end{verbatim}

\verb'qualident' is a Modula-2 type being defined, \verb'size'
is its size in bytes, and \verb'Attr' is a family to which it belongs:

\verb'-M2TYPE = SYSTEM.INT16 = 2, SIGNED'

See \ref{using:modrules:mapping} for more information.

\ifonline \else
\end{description}
\fi


