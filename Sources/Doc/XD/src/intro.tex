\chapter{Introduction}
\pagenumbering{arabic}

From its first release in 1991, the \XDS{}\footnote{Previous versions or XDS were
known as Extacy and OM2} family of \mt{}/\ot{} development systems
provided a set of basic debugging facilities, such as execution history
and profiling. When it came to symbolic debugging, however, it relied on
an "underlaying" C compiler package. Nothing was wrong about it, as long
as \XDS{} compilers were implemented as translators to ANSI C.

For owners of the first native code XDS compilers for Intel x86 line of
processors, launched in the second half of 1996, having a C compiler
is no longer essential. Nevertheless, XDS understands that for any
serious development a symbolic debugger is an absolute requirement.

The {\bf \XDS{} Debugger} (XD) pretends to meet this requirement. It is
a full source level symbolic debugger which provides common debugging
facilities, such as execution in step mode, breakpoints and breaks setting,
data examination and modification, etc.

The debugger may be used in two modes, dialog and batch. The dialog mode
is interactive, while in the batch mode XD may automatically
execute a debugging scenario (or a number of scenarios) described in
special control files, and log the scenario execuion results.
This allows the debugger to be used to automate testing.

\section{New in version 1.1}

The following changes and improvementes have been made to XD since 
version 1.0:

\begin{itemize}
\item \See{Improved locators}{}{dialog:intro:info}
\item 16-bit character type and 64-bit integer type (Java \verb'char' and 
      \verb'long int') are partially supported
\item Exceptions generated by the \XDS{} run-time library (type guards, 
      array index checks, etc.) are now detected
\item The program may be restarted at entry point
      (see \ref{dialog:executing:entry})
\item A counter may be turned into a delayed breakpoint 
      (see \ref{dialog:breaks:delayedsticky})
\item The \See{{\bf Float Registers} window}{}{dialog:data:float}
\item Constants and symbols are displayed according to rules of the respective
      language (e.g. a null pointer is displayed as \verb'NIL' in \mt{} 
      variables and as \verb'null' in Java variables)
\item Global variables of any module can be displayed in a separate window
      (see \ref{dialog:navigating:modules}, \ref{dialog:data:module})
\item Local variables and parameters from any procedure's stack frame 
      can be displayed in the 
      \See{{\bf Local variables} window}{}{dialog:data:locals}
\item A memory dump may be saved to file (see \ref{dialog:data:dump})
\item Watch expressions are saved in session files
\item Detection of the actual dynamic type of an Oberon-2 pointer 
      or a Java object (see \ref{dialog:options})
\item Debug scenario record and playback (see \ref{dialog:scenarios})
\item Symbolic names may be used instead of numbers to identify breaks in 
      the batch mode (see \ref{batch:BREAK})
\item All types of breakpoints may now be set in the batch mode 
      (see \ref{batch:BREAK})
\item In the batch mode, breakpoints may be set at a procedure's 
      prologue, body, epilogue, or return instruction 
      (see \ref{batch:BREAK})
\item Command-line arguments may now be referenced in control files
      (see \ref{expr:builtin:arg})
\item \See{Remote debugging}{Chapter }{remote} is now supported
\item Support for XD has been added to the \See{HIS utlity}{Appendix }{his}
\end{itemize}

