\chapter{About \XDS}
\pagenumbering{arabic}

\section{Welcome to \XDS}

\ifonline \xds{}\else $\mbox{XDS}^{\mbox{\tiny TM}}$\fi{} is a family name
for professional \mt{}/\ot{} programming systems for 
Intel x86-based PCs (Windows and Linux editions are available).
\ifgenc XDS-C is a "via C" cross-compiler that allows you
to target virtually any system, from embedded to Unix servers.
\fi
\xds{} provides an
uniform programming environment for the mentioned
platforms and allows design and implementation of portable software.

The system contains both \mt{} and \ot{} compilers. These
languages are often called {\bf ``safe''} and {\bf ``modular''}. The
principle innovation of the language \mt{} was the module concept,
information hiding and separate compilation.

\ot{} is an object-oriented programming (OOP) language based on
\mt. With the introduction of object-oriented facilities,
extensible project design became much easier. At the same time, \ot{} is
quite simple and easy to learn and use, unlike other OOP
languages, such as C++ or Smalltalk.

The \xds{} \mt{} compiler implements ISO 10514 standard of \mt{}.
The ISO standard library set is accessible from both \mt{} and \ot{}.

\xds{} is based on a platform-independent front-end for both source
languages which performs all syntactic and semantic checks on the
source program. The compiler builds an internal representation of
the compilation unit in memory and performs platform-independent
analysis and optimizations. After that the compiler emits output
code. It can be either native code for the target platform or
text in the ANSI C language. ANSI C code generation allows you to
cross compile \mt{}/\ot{} for almost any platform. 

Moving to a new language usually means throwing away or rewriting
your existing library set which could have been the work of many
years. \xds{} allows the programmer to mix \mt{}, \ot{}, C and
Assembler modules and libraries in a single project.

\xds{} includes standard ISO and PIM libraries along with a set
of utility libraries and an interface to the ANSI C library set.

\ifgenc

 \xds{} compilers produce optimized ANSI C code which is further
 compiled by a C compiler. The call of a C compiler can be done
 transparent for the user. However, \xds{} can be used as
 \mt/\ot{} to C translator, as it produces easy readable and
 understandable text. It is also possible to preserve your source
 code comments in their original context.

\fi
\ifgencode
   \xds{} compilers produce highly optimized 32-bit code and debug
   information in the
   \iflinux {\bf stabs} format.\else {\bf Codeview} and {\bf HLL4} formats.\fi{}
 \ifwinnt
   It is possible to access the Win32 API from both \mt{} and \ot{}
   programs with the help of supplied defition modules.
 \fi
 \iflinux % !!! In fact, if Unix
   Definition modules for the POSIX API and the entire X Window/Motif API
   are included in the XDS distribution package.
 \fi

\fi % \ifgencode

\section{Conventions used in this manual}

\subsection{Language descriptions}

Where formal descriptions for language syntax constructions appear, an
extended Backus-Naur Formalism (EBNF) is used.

These descriptions are set in the {\tt Courier} font.

\verb'    Text = Text [ { Text } ] | Text.'

In EBNF, brackets "\verb'['" and "\verb']'" denote optionality of the
enclosed expression, braces "\verb'{'" and "\verb'}'" denote repetition
(possibly 0 times), and the vertival line "\verb'|'" separates
mutually exclusive variants. % ??? What about parenthesis.

Non-terminal symbols start with an upper case letter (\verb'Statement').
Terminal symbols either start with a lower case letter (\verb'ident'), or are
written in all upper case letters (\verb'BEGIN'), or are enclosed within
quotation marks (e.g. \verb'":="').

\subsection{Source code fragments}

When fragments of a source code are used for examples or appear within
a text they are set in the {\tt Courier} font.

\begin{verbatim}
    MODULE Example;

    IMPORT InOut;

    BEGIN
      InOut.WriteString("This is an example");
      InOut.WriteLn;
    END Example.
\end{verbatim}

