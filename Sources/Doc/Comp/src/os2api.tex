\chapter{Using the OS/2 API}
\label{os2api}

This chapter contains a short description of the OS/2 API usage
in your Modula-2/Oberon-2 programs.

\section{Obtaining API documentation}

Your XDS package includes the OS/2 API definition module and the import
library, but no documentation on the API. You have to use
third-party books or the on-line documentation from the
IBM Developer's Toolkit for OS/2, which is available on
Developers Connection CD-ROMs and along with some C compilers,
for instance, IBM VisualAge C++ or Watcom. It can also be downloaded
from the IBM Developers Connection Web site. % !!! URL?

Unfortunately, we can not afford purchasing the on-line API
documentation sources from IBM for adapting them to Modula-2
and distributing with XDS. We regret the inconvenience.

\section{Using the OS/2 API in your programs}

The whole API is defined in a single Modula-2 definition module \verb'OS2.DEF',
which resides in the \verb'DEF\OS2' subdirectory of your \XDS{} installation.
It is very large (about 20000 lines) and its symbolic file would be very large too
and would create a great time and memory overhead during compilation.
Fortunately, the original C header files were built using conditional
compilation sections --- a macro, for instance \verb'INCL_DOSPROCESS', has to be
defined before including the main \verb'OS2.H' header file to select the
appropriate section. This idea, as well as constant names, is preserved
in XDS.

Thus, to use an OS/2 API call in your program:

\begin{enumerate}

\item Declare the correspondent \verb'INCL_xxx' option in the PROJECT file:

\verb'    -INCL_DOSPROCESS:+'

\item Add the module \verb'OS2' to the import list of your module:

\verb'    IMPORT OS2;                 (* qualified import *)'

or

\verb'    FROM OS2 IMPORT DosBeep;    (* unqualified import *)'

\item Insert the required call:

\verb'    OS2.DosBeep(1000,100);      (* qualified import *)'

or

\verb'    DosBeep(1000,100);          (* unqualified import *)'

\end{enumerate}

{\bf Note:} You have to manually recompile the \verb'OS2.DEF' file after you
add or remove \verb'INCL_xxx' options from your project. Alternatively,
you may either remove the \verb'OS2.SYM' file or use the =all submode
when making your project.

If your computer has enough resources to use the complete \verb'OS2.SYM'
file, compile the \verb'OS2.DEF' file with the following options:

\verb'    -INCL_BASE:+ -INCL_PM:+'

then copy the resulting \verb'OS2.sym' file to the
\ifgencode \verb'SYM\X86' \else \verb'SYM\C' \fi
subdirectory of your XDS installation.
Remove \verb'"$!/../def/os2"' reference from your master redirection file, % !!! \ref{}
\ifgencode \verb'BIN\xc.red'\else\verb'BIN\xm.red'\fi,
and redirection file template,
\ifgencode \verb'BIN\xc.ret'\else\verb'BIN\xm.ret'\fi

\subsection{Building PM applications}

If your program is a Presentation Manager application, toggle
the PM option ON in the project file:

\verb'    +PM'

\subsection{Declaring callbacks}

If you have to pass a procedure as a parameter to an API call (a window
procedure, for example), you should specify the "SysCall" calling
and naming convention (see \ref{multilang:direct}):

\begin{verbatim}
    PROCEDURE ["SysCall"] MyWindowProc
    ( hwnd         : OS2.HWND;
      msg          : OS2.ULONG;
      mp1          : OS2.MPARAM;
      mp2          : OS2.MPARAM
    )              : OS2.MRESULT;
\end{verbatim}

\subsection{Additional macros}

In order to reduce need for type casting, the following "macros" are
provided in the definition module \verb'OS2'
(they are implemented in the run-time library):

\begin{verbatim}
    MPFROMUSHORT    USHORT1FROMMP
    MPFROM2USHORT   USHORT2FROMMP
    MPFROMULONG     ULONGFROMMP

    MRFROMUSHORT    USHORT1FROMMR
    MRFROM2USHORT   USHORT2FROMMR
    MRFROMULONG     ULONGFROMMR
\end{verbatim}

Although these "macros" are declared in the module \verb'OS2', you are
requested to import the module \verb'OS2RTL' as well. More precisely,
the original IBM OS/2 Toolkit header files, which are used during C compilation
in case of XDS-C, do {\em not} contain these macros, so they are implemented in
the \verb'OS2RTL.h' header file supplied with XDS-C.
To have that header included into the generated C code, you need to
import the module \verb'OS2RTL'.
In XDS-x86, these macros are implemented as procedures which reside
in the XDS run-time library, so they will be linked in even if you do not
import \verb'OS2RTL', but importing it will ensure portability between
XDS-x86 and XDS-C.

\section{Using resources}

\subsection{Automatic resource binding}

In early XDS versions, resources had to be bound to an executable
"manually", i.e. by invoking a resource compiler, each time after linking.
Starting from version 2.20, it is possible to specify any file in
the \See{{\tt!module} project file directive}{}{xc:project}
and iterate files by extensions in \See{makefile templates}
{}{xc:template}.

\ifcomment !!!
On the other hand, since LINK386 does not accept resource files,
it is required to call RC separately which is impossible in the
make system currently implemented in XDS compilers. To overcome
this problem, a fake linker called XLINK, which invokes both
LINK386 and RC, was added to the package. Its response file has
the following format:

    <LINK386 responce file> [
    <empty line>
    { <binary resource file name> } ]

The default template file, XC.TEM, produces response file in this
format. For example:

    /PM:PM /NOLOGO /BAT /NON /NOI /NOO /ST:100000 app.obj,
    app.exe,
    nul,
    c:\xds\lib\x86\libxds.lib+
    c:\xds\lib\x86\os2face.lib

    d:\work\app.res
\fi

Thus, if your application uses resources, you may specify
binary resource files as modules in your project to have them
been automaticlly bound to the executable, provided that
the used \See{template}{}{xc:template} has adequate support:

\verb'    !module myapp.res'

\subsection{Header files}

XDS package contain no resource-related tools, such
as a dialog editor or a resource compiler. By now, you have to use
programs which come with OS/2 (RC), the Developer's Toolkit, or a
C compiler. Those programs, however, have no idea about what
Modula-2 is and work only with C header files. Therefore,
constants which identify resources have to be defined in
both C headers and Modula-2 definition modules, which may
easily go out of sync.

To solve this problem, we suggest you to define resource
identifiers in header files and then use H2D to convert these
files to definition modules before compilation.

\section{PM application template}

The \verb'SAMPLES\TEMPLATE' subdirectory of your installation contains
a Presentation Manager application "template" which you can use as
a base for your own applications. The sample program uses the
standard menus and dialogs that most applications would use.
The source is designed to serve as a base for any Presentation
Manager application and was written in such a way that it can be
easily modified to handle any application specific commands.

The template PM program was taken from the IBM Developer's Toolkit for
OS/2 Warp and translated to Modula-2.

