% !!! Type designators: SIZE(T^)
% !!! BiosIO, MsMouse, FIOR descriptions.

\chapter{TopSpeed Compatibility Pack}
\label{tscp}

The TopSpeed \mt{} compiler for DOS and OS/2 by Jensen and Partners
International had been rather popular in late 1980s. There are millions of 
lines of TS \mt{} code still used in various environments. Unfortunately, 
since that time the world has moved to Windows and Linux, and TopSpeed users
could only watch this process as their compiler is no longer available as
a standalone product, and is not actively developed, marketed, and supported
by its new owners. 
So there is a need for them to simultaneously change the compiler and the
target operating system.

\xds{} \mt{} compilers, available for Windows and Linux, are likely
to be considered as a replacement for TopSpeed. But moving from TopSpeed
to \xds{} can be difficult: the JPI implementation of the language is
based on PIM, whereas \xds{} follows the ISO 10514 standard, which is not
a superset of PIM. In addition, both compilers provide different sets of
language extensions.

That's why the TopSpeed Compatibility Pack (TSCP) has been developed by 
Excelsior. The XDS compiler which is part of TSCP has one additional option ---
\OERef{TOPSPEED}. Setting it ON enables a set of 
language extensions that makes the compiler more compatible with TopSpeed.
These extensions are described in the following sections.

TSCP also contains a \See{set of TopSpeed-like library modules}{}{tscp:lib}.

\section{Conditional compilation}
\label{tscp:cc}

The existing users of your DOS and/or OS/2 software products are not likely
to move to Win32 at once as you do it. So it is desirable to be able to 
compile the same sources with both TopSpeed and \xds{}. To facilitate this,
\xds{} supports TopSpeed conditional compilation syntax.

{\bf Note:} In TopSpeed, an identifier used in a conditional compilation 
clause must be a boolean constant, either predefined, defined in the project 
file, or in the source text. In \XDS{}, it has to be an option name.

The \verb'__XDS__' option is always set to \verb'TRUE' in \XDS{} \mt{}, so
we recommend you to define the \verb'__XDS__' constant in your TopSpeed project
file:

\verb'    #pragma define(__XDS__=>off)'

and then use the above scheme to isolate TS- and \XDS{}-specific portions of
code. See example in section \ref{tscp:pragmae:syntax}.


\section{Declarations and visibility}

\subsection{Alias declaration}

    TopSpeed alias declaration (\verb'CONST id1 ::= id2;') is available in
    \XDS{} if the option \OERef{TOPSPEED} is turned ON.

\section{Types}

\subsection{Numeric types}

    Types \verb'LONGCARD', \verb'LONGINT', \verb'SHORTCARD', and 
    \verb'SHORTINT' are not part of ISO \mt{}, but they are available
    in \xds{} \mt{} if the \OERef{M2ADDTYPES} compiler option is switched on.
    Usage of types \verb'SYSTEM.INT8', \verb'SYSTEM.INT16', etc. may 
    also be considered.

    TopSpeed \mt{} types \verb'TEMPREAL' and \verb'FIXREAL' have no analogs
    in \xds{}.

\subsection{Sizes of base types}

    \xds{} compilers generate 32-bit code, so the base \mt{}
    types, such as \verb'CARDINAL', \verb'INTEGER', and \verb'BITSET',
    are 32-bit by default, whereas in TopSpeed \mt{} these types are 16-bit.
    To achieve better performance, it is recommended to use 32-bit types 
    anywhere except in declarations that represent externally defined 
    data structures, such as network packet layouts or database records. 
    Unfortunately, this requires a lot of analysis that can only be done
    manually and requires deep understaning of the program being migrated,
    and therefore is error prone. This problem resembles the famous Year 2000 
    problem --- no one can tell where in a million of source lines your 
    program relies on base types being exactly 16 bits wide.

    You may consider usage of the following techniques that do not require 
    this analysis:
    \begin{itemize}
    \item Turn the \OERef{M2BASE16} option ON in your project files. This would 
          force base types to be 16-bit. Types \verb'SHORTINT', \verb'INTEGER',
          and \verb'LONGINT', as well as \verb'SHORTCARD', \verb'CARDINAL',
          and \verb'LONGCARD', are assignment and parameter compatible. 
          The compiler inserts the required conversions and checks 
          automatically. You may need to modify your code if there are
          any calls of library procedures that accept or return \verb'BITSET's.

          Take into account that the \xds{} runtime library has been 
          compiled with the \OERef{M2BASE16} option turned OFF. 

    \item Redefine these types in all your modules using fixed size types 
          from the module SYSTEM:
\begin{verbatim}
    (*%T __XDS__ *)
      TYPE 
        INTEGER  = SYSTEM.INT16;
        CARDINAL = SYSTEM.CARD16;
        BITSET   = SYSTEM.SET16;
    (*%E *)
\end{verbatim}
    \end{itemize}

\subsection{Based and virtual pointers}

    There are no based and virtual pointers in \xds{}.

\subsection{Compatibility with system types }

    In TopSpeed \mt{}, types \verb'BYTE', \verb'WORD', and \verb'LONGWORD'
    are assignment compatible with any type of corresponding size
    (1, 2, or 4 bytes), and allow testing for equality or inequality.
    Due to a far more complex definition of the \verb'BYTE' and \verb'WORD'
    types in ISO \mt{}, the only operation directly defined for them is an
    assignment and there are special rules affecting parameter compatibility
    (see \xds{} User Guide, System parameter compatibility for further details).
    The \verb'LONGWORD' type does not exist in \xds{}.

\section{Objects and values}

\subsection{Fixed variable placement}

    In TopSpeed, a variable can be placed at a fixed physical address 
    which is specified in segment:offset form:

\verb'    VAR Screen [0B800H:0] : ARRAY [1..25][1..80] OF Cell;'

    In \XDS{}, the \verb'SYSTEM.MAKEADR' function has to be used:

\verb'    VAR MyCardIO [SYSTEM.MAKEADR(0FFFF0000H)] : BITSET;'

\subsection{Value constructors}

    In ISO \mt{}, constructors of array type and record type value use curly
    braces (\verb'{}') rather than parenthesis (\verb'()').

\begin{verbatim}
    TYPE
      Struct = RECORD
                 date :CARDINAL;
                 time :CARDINAL;
               END;

    CONST
      (* s = Struct ( 10, 12 );   TS manner *)

      s = Struct { 10, 12 };  -- XDS (ISO) manner
\end{verbatim}

\section{Expressions}

\subsection{Function calls in designators}

    If the option \OERef{TOPSPEED} or the option \OERef{M2EXTENSIONS} is set ON,
    \xds{} recognizes function procedure calls in designators, including
    predefined, SYSTEM, and type transfer procedures:

\begin{verbatim}
    CurPos(point[i,j])^.x
    GetCurDate().year
    REC_PTR(adr)^.field
\end{verbatim}

\subsection{Logical shift operators}

    The '\verb'<<'' and '\verb'>>'' infix operators of logical
    left/right shift, defined for cardinal types in TS, are available in
    \XDS{} if the option \OERef{TOPSPEED} or the option \OERef{M2EXTENSIONS} is 
    set ON.

\subsection{Comparison of structured values}

    TopSpeed \mt{} supports comparison of structured values (arrays and 
    records) as a language extension. In general, alignment and 
    packing rules of the target environment may cause existence of 
    gaps between elements of an array or fields of a record. 
    So comparison of structured values in raw mode (e.g. using 
    \verb'Lib.Compare') is likely to produce incorrect results.
    In ISO \mt{}, you have to compare stuctured values element by element 
    or field by field,


\section{Statements}

\subsection{CASE without ELSE}

In TopSpeed \mt{}, absence of the \verb'ELSE' clause in a \verb'CASE'
statement is equivalent to presense of an empty \verb'ELSE' clause.
In \xds{} \mt{}, according to ISO 10514, if the value of the case
selector is not contained in any case label list and the ELSE clause
is not present, an exception shall be raised. Therefore, the \verb'CASE'
statement in the following code fragment:

\begin{verbatim}
    x := 2;
    CASE x OF
    |0: p0;
    |1: p1;
    END;
\end{verbatim}

does nothing if compiled by TopSpeed, and raises an exception
if compiled by \xds{}.

The recommended solution is to add empty \verb'ELSE' clauses to all your
\verb'CASE' statements that lack it, as this would make no difference in 
TopSpeed. To quickly find all such statements, turn on warning 319:

\begin{verbatim}
    C:\WORK>xc t.mod -woff319-'

    XDS Modula-2 v2.40 [x86, v1.08] - build 20.10.1998
    Compiling "t.mod"

    * [t.mod 12.05 W319]
    * CASE statement without ELSE
\end{verbatim}

\subsection{FOR loop control variable}

    ISO \mt{} requires that a \verb'FOR' loop control variable must be declared
    in the scope to which the loop belongs. TopSpeed does not make this
    restriction, so you may need to redesign some of your FOR loops.

\subsection{GOTO statement}

    There are no \verb'GOTO' and \verb'LABEL' clauses in ISO \mt{}. \XDS{}
    provides them as an extension if the option \OERef{TOPSPEED} is turned ON.

\section{Procedures}

\subsection{External procedures}

     \xds{} compilers allow only foreign procedures to be declared as 
     external.

     A declaration of an external procedure consists of a procedure header 
     only, with the procedure name preceded by the symbol "\verb'/'":

     \verb'    PROCEDURE ["C"] / putchar(ch: SYSTEM.CARD8): SYSTEM.INT32;'

\subsection{Open arrray parameter compatibility}

    In TopSpeed \mt{}, an expression of type \verb'T' can be substituted as 
    a parameter whose formal type is \verb'ARRAY OF T'. \xds{} supports 
    this if on of the \OERef{M2EXTENSIONS} or \OERef{TOPSPEED} is turned ON.

\subsection{Procedure types}

    In ISO \mt{}, a procedure belongs to a particular procedure type
    if the types of each formal parameter, taken in turn, are exactly the same.
    In TopSpeed \mt{}, there is one exception: if one of the two types is 
    \verb'SYSTEM.ADDRESS', another type can be any pointer type,
    so the following piece of code is legal:

\begin{verbatim}
    TYPE
      SetNameMethod = PROCEDURE(SYSTEM.ADDRESS,ARRAY OF CHAR);

      NodePtr = POINTER TO Node;
      Node = RECORD
               name   : ARRAY [0..63] OF CHAR;
               setname: SetNameMethod;
             END;
                 
    PROCEDURE SetNodeName(p: NodePtr; name: ARRAY OF CHAR);
    BEGIN
      Str.Copy(p^.name, name);
    END SetNodeName;

      .  .  .
   
     NEW(node);
     node^.setname := SetNodeName;
\end{verbatim}

    In TopSpeed \mt{}, a procedure type definition may reference an incomplete
    type containing that definition, provided that the incomplete type has 
    already been referenced:

\begin{verbatim}
    TYPE
      P = POINTER TO R;
      R = RECORD
            data: CARDINAL;
            proc: PROCEDURE(R);
          END;
\end{verbatim}

    \xds{} does not allow this, so you have to define the procedure type 
    separately:

\begin{verbatim}
    TYPE
      P = POINTER TO R;
      R_proc_type = PROCEDURE(R);
      R = RECORD
            data: CARDINAL;
            proc: R_proc_type;
          END;
\end{verbatim}

    The \verb'NULLPROC' value is absent in \xds{}; \verb'NIL' should be
    used instead. You may consider declaring \verb'NULLPROC' under 
    conditional compilation in modules that use it:

\begin{verbatim}
    (*%T __XDS__ *)
      CONST 
        NULLPROC = NIL;
    (*%E *)
\end{verbatim}

\subsection{Predefined procedures}

     Predefined procedures \verb'FarADR', \verb'NearADR', \verb'Ofs', and 
     \verb'Seg' are specific to the 80x86 16-bit real and protected modes
     and therefore are not available in \XDS{}.

     The \verb'FieldOfs' and \verb'VSIZE' procedures are predefined if the
     option \OERef{TOPSPEED} is set ON and are available from the module SYSTEM
     otherwise.

\subsection{Type conversion vs type cast}

     In TopSpeed \mt{}, the predefined function procedure 
     \verb'VAL(type, expr)' performs {\em type conversion} if both \verb'type'
     and \verb'expr' are numeric or ordinal, and performs {\em type cast}
     otherwise. In ISO \mt{}, \verb'VAL' may only be used in the former case;
     \verb'SYSTEM.CAST' has to be used for type casting.

     According to PIM, a {\em type transfer function} \verb'type(expr)' shall 
     not affect the bit pattern representaion of the result of \verb'expr'.
     Unfortunately, TopSpeed does not follow this specification and treats
     \verb'type(expr)' exactly as \verb'VAL(type,expr)' (see above).
     Due to introduction of such ambiguities by pre-ISO compiler vendors, 
     this feature was not included in the International Standard. 
     \xds{} provided it as a language extension (option 
     \OERef{M2EXTENSIONS} ON), but treated it as \verb'SYSTEM.CAST(type,expr)'.
     Now, if the \OERef{TOPSPEED} option is turned ON, \xds{} will treat this
     construction exaclty as TopSpeed. This ambiguity may lead to porting 
     problems that are hard to detect, so the compiler issues a warning
     upon encountering a type transfer function.
     It is recommended to use the built-in \verb'VAL' and \verb'SYSTEM.CAST' 
     function procedures for type conversion and type cast respectively. 

\subsection{Inline and code procedures}

    In \xds{}, there is no way to point out an inline procedure explicitly 
    (yet), however native-code compilers can make inline substitutions of 
    some procedures automatically. It is also not possible to define inline
    procedures in a definition module. You have to move such procedures
    into implementation modules.

    As for TopSpeed binary inline procedures, they also have no equivalent 
    in \xds{}. Since they contain 16-bit mode instructions, they would have
    to be rewritten anyway. \xds{} inline assembler may be used to implement 
    machine code routines.

\section{Modules}

\subsection{Import}

    In TopSpeed, any entity imported into a definition module is "automatically" imported
    into the corresponding implementation module. In ISO \mt{}, all entities
    required by an implementation module should be explicitly imported into it.

    TopSpeed \mt{} supports qualification of entities defined in a definition
    module in the respecive implementation module. In \XDS{}, this is not 
    possible.

\subsection{Hidden pointer base types}

    In TopSpeed, if \verb'T' is a type defined in an implementation module,
    it is still possible to declare a type \verb'POINTER TO T' in a 
    definition module:

\begin{verbatim}
    DEFINITION MODULE T;
    
    TYPE 
      p1 = POINTER TO t1;
      r1 = RECORD
             p2: POINTER TO t2;
           END;
    
    END T.
\end{verbatim}

    \xds{} does not allow this, so you have to either
    \begin{itemize}
    \item define such types as opaque
    \item move base types into definition modules
    \end{itemize}

\subsection{Export from local module}

    In TS. the keyword \verb'QUALIFIED' (in local modules) has no effect and
    a name of an object exported from a local module can be used with or
    without qualification. In ISO \mt{}, there is a distinction between
    qualified an unqualified export. You will have to introduce uniformity
    here: if you mostly use objects exported from local modules with 
    qualification, ensure presence of the \verb'QUALIFIED' keyword, otherwise
    remove it. Compilation errors will then indicate places where you have to
    add or remove qualification.

\subsection{The module SYSTEM}

    In ISO \mt{}, all objects from the module \verb'SYSTEM', such as types
    \verb'ADDRESS', \verb'WORD', \verb'BYTE', and the procedure \verb'ADR'
    have to be explicitly imported. In TopSpeed \mt{}, they are predefined.

    Although the \xds{} module \verb'SYSTEM' contains procedures 
    \verb'NEWPROCESS', \verb'TRANSFER', and \verb'IOTRANSFER', 
    it is recommended to replace them with more powerful mechanisms provided 
    by the ISO system module \verb'COROUTINES'. See also \ref{mt}.

    The set of platform-specific facilities is almost completely different
    in TopSpeed and \xds{} variants of the module \verb'SYSTEM'.

\section{Intel x86 16-bit mode specifics}

    TopSpeed \mt{} supports Intel x86 16-bit real and protected modes,
    whereas \XDS{} produces code for the 32-bit flat memory model. 
    Therefore, code that deals with segment:offset addresses has to be
    rewritten for \XDS{}.

    In \XDS{}, there is no analog for TopSpeed pointer constructors of 
    the form \verb'[seg:ofs type]'. These constructors are commonly used 
    to directly interface to hardware and to access DOS and BIOS data areas,
    so routines that utilize this feature have to be redesigned anyway,

    The following TopSpeed \mt{} types and procedures, specific to
    Intel x86 16-bit real and protected modes, are absent in \xds{}:

    \verb'    FarADDRESS   FarNIL   FarADR' \\
    \verb'    NearADDRESS  NearNIL  NearADR' \\
    \verb'    Ofs Seg'

    There are so called based pointers in TopSpeed, i.e. pointers belonging to
    a particular segment. This feature is useless in 32-bit flat mode, so
    it is not available in \xds{}.

\section{Object oriented extensions}

    There are no object-oriented extensions in \xds{} \mt{}. Although \xds{}
    features \ot{} as well, its OO facilities are dissimilar in many
    aspects from provided in TopSpeed \mt{}. So, unfortunately,
    the required porting efforts for a program that makes heavy use of
    TS OO extensions may be unaffordable.


\section{Pragmae}

\subsection{Pragma syntax}
\label{tscp:pragmae:syntax}

    \xds{} pragma syntax is completely different from TS. Most TopSpeed 
    pragmae have no analogs in \XDS{}, and many of them are pretty useless
    in the 32-bit flat memory model. 
    \See{Conditional compilation}{}{tscp:cc}
    has to be used to isolate TopSpeed and \XDS{} pragmae:

\begin{verbatim}
    (*%T __XDS__ *)
      <* PUSH *> <* COVERFLOW - *>
    (*%E *)
    (*%F __XDS__ *)
      (*# save, check(overflow=>off) *)
    (*%E *)
      hash := hash*alpha+ORD(ch);
    (*%T __XDS__ *)
      <* POP *>
    (*%E *)
    (*%F __XDS__ *)
      (*# restore *)
    (*%E *)
\end{verbatim}

\subsection{Open array parameters passing}

    When an array is passed as a value parameter in TS, it can be either copied
    onto the stack or passed by reference, depending on the \verb'call(o_a_copy)'
    pragma setting. In \xds{}, an array is always copied onto the stack unless the
    formal parameter is marked as read-only, for instance:

\verb'    PROCEDURE PrintStr (s  : ARRAY OF CHAR);  -- pass by value'

\verb'    PROCEDURE PrintStr (s- : ARRAY OF CHAR);  -- pass by reference'

    Note: the \OERef{M2EXTENSIONS} option should be turned on.

    When a procedure parameter has type \verb'ARRAY OF CHAR', TopSpeed allows an
    actual parameter of type \verb'CHAR' to be passed. This is not allowed by
    the ISO standard, and \xds{} will not accept it.

\subsection{Bitwise operations on cardinals}

    In TopSpeed \mt{}, the pragma \verb'option(bit_expr=>on}' enables
    bitwise operations \verb'AND', \verb'OR', and \verb'NOT' to be applied
    to cardinals. In \XDS{}, this feature is enabled if the \OERef{M2EXTENSIONS}
    option is set ON.

\subsection{Other pragmae}

    More information on TS-to-XDS pragmae mapping will be provided in future releases.

\pagebreak

\section{\xds{} TS-like library}
\label{tscp:lib}

Table \ref{tscp:matrix} shows which TopSpeed-like library modules are implemented
in which edition of TSCP.

In order to preserve the overall level of the \xds{} libraries portability,
a number of modifications was made to the original interface and functionality of the
modules. Also, some restrictions were induced by differences between target
operating systems. There are also some useful extensions. See the following sections
for details.

{\bf Note:} The library is compiled with the \OERef{M2BASE16} compiler option
turned OFF, so types \verb'INTEGER', \verb'CARDINAL', and \verb'BITSET' 
are 32-bit.

\begin{table}[htbp]
\begin{center}
\begin{tabular}{|l|c|c|c|c|}
\hline
\bf Module & XDS-x86     & XDS-x86  & XDS-C \\
\bf name   & (Windows)   & (Linux)  & (all platforms) \\
\hline
\tt BiosIO     & $\surd$ & $\surd$  & $\surd$ \\
\tt FIO        & $\surd$ & $\surd$- & $\surd$- \\
\tt FIOR       & $\surd$ & $\surd$  & $\surd$ \\
\tt Graph      & $\surd$ & -        & - \\
\tt IO         & $\surd$ & $\surd$  & $\surd$ \\
\tt Lib        & $\surd$ & $\surd$  & $\surd$ \\
\tt MATHLIB    & $\surd$ & $\surd$  & $\surd$ \\
\tt MsMouse    & $\surd$ & $\surd$  & - \\
\tt ShtHeap    & $\surd$ & $\surd$  & $\surd$ \\
\tt Str        & $\surd$ & $\surd$  & $\surd$ \\
\tt Window     & $\surd$ & $\surd$  & - \\
\hline
\end{tabular}
\caption{Module/product availability matrix}
\label{tscp:matrix}
\end{center}
\end{table}

%-------------------------------------------------------------------------------

\subsection{Module FIO}

{\bf MODIFICATIONS}

\begin{enumerate}

\item Since the module {\bf FIO} is implemented on top of the ISO standard library,
      the procedural variable \verb'RunTimeError' has been removed. If \verb'IOCheck'
      is set to \verb'TRUE', an I/O error raises an exception, so ISO Modula-2
      exception handling facilities can be used. Exceptions raised in {\bf FIO}
      can be indentified in an exception handler by means of the procedures
      \verb'FIO.IsFIOException()' or \verb'IOChan.IsChanException()'.

\item The constant \verb'MaxOpenFiles' is no longer needed, since the maximum number of
      open files is limited only by the system resources. If there are
      too many open files, either an exception will be raised or the error
      \verb'tooManyOpen' will be returned by the procedure \verb'FIO.IOresult()'
      (it depends on the state of the variable \verb'IOcheck').

\item File share modes are defined as an enumeration type:

\begin{verbatim}
    TYPE
      SHAREMODE = (ShareCompat,
                   ShareDenyRW,
                   ShareDenyRD,
                   ShareDenyWR,
                   ShareDenyNone);
\end{verbatim}

      Supported target operating systems do not distingiush between subsequent opens
      of a file done by the same or different process, so \verb'ShareCompat' works
      as \verb'ShareDenyWR' in \verb'OpenRead' and as \verb'ShareDenyRW' in other
      file open procedures.

\item Standard file handles (e.g. \verb'StandardInput') are predefined variables of
      the \verb'File' type rather than constants.

\item Type of the buffer parameter for \verb'RdBin' and \verb'WrBin' has been changed to
      \verb'ARRAY OF SYSTEM.LOC' which works perfectly the same as \verb'ARRAY OF BYTE'.

\item When a file opening error occurs, the procedures \verb'Open', \verb'OpenRead',
      \verb'Create', etc. return a special invalid value instead of MAX(CARDINAL).
      The procedure \verb'InvalidFile' always returns that value.

\item A disk label can be obtained by calling \verb'FIO.GetLabel' rather
      than \verb'FIO.ScanFirst' with the volume attribute set.

\item After directory scanning (\verb'FIO.ReadFirstEntry', \verb'FIO.ReadNextEntry') process
      is completed, a call to the \verb'FIO.ReadClose' procedure {\em has to be issued} with
      the scan-buffer passed as a parameter.

\item The procedure \verb'FIO.SetDrive' does not return the total number of drives
      available on the system.

\item The \verb'EOF' flag is not set when the current read position {\it reaches} end of file,
      but on attempt to read {\it past it}, as in ISO routines.

\item The \verb'Ctrl-Z' character is {\it not} recognized as an end-of-file mark.

\item For all procedures accepting or returning a drive number, drive numeration
      starts from 1 (1 is drive A, 2 is drive B and so on).

\item The {\bf SysErr} module contains constants defining symbolic names of error
      codes. Those constants may be used to test a return code obtained from
      \verb'FIO.IOresult'. The constant \verb'SysErr.allRight' corresponds to a normal
      procedure termination (it is recommended to use this constant instead of
      zero value).

\item 32-bit file length is supported.

\item A file handle obtained from a system call can {\bf NOT} be directly passed
      to {\bf FIO} procedures. It has to be converted to a value of the type \verb'File'
      using the \verb'AppendHandle' procedure:

\begin{verbatim}
    PROCEDURE AppendHandle(fh : LONGCARD
                          ;ReadOnly : BOOLEAN
                          ) : File;
\end{verbatim}

\item The type \verb'FIO.FileStamp' is declared as equal to 
      \verb'SysClock.DateTime'.

\item The parameter of \verb'FIO.Close' now is a VAR parameter and the procedure
      shall assign an "invalid" value to it if the close operation succeeds,

\end{enumerate}

{\bf RESTRICTIONS}

\begin{enumerate}
\item There is no multithread program support yet. Namely, the \verb'IOresult',
      \verb'ThreadEOF' and \verb'ThreadOK' procedures are not implemented yet,
      Nevertheless, the module can be used in multithread programs with the
      reservations listed above.

\item File share modes are only supported by Native XDS-x86 for Windows.

\item Assinging a buffer to a file is not implemented yet. I/O operations are
      buffered internally, but there is no way to flush an internal buffer (yet).

\item The procedures \verb'GetStreamPointer' and \verb'AppendStream' 
      have been excluded,

\item The predefined variable \verb'PrinterDevice' is not eliminated yet.

\item \verb'RdReal' and \verb'RdLngReal' do not recognize exponent-only zeroes,
      such as \verb'E+0'.
\end{enumerate}

{\bf PRODUCTS}

Native XDS-x86 for Windows\\
Native XDS-x86 for Linux (with limitations)\\
XDS-C (with limitations)

%------------------------------------------------------------------------------

\subsection{Module Graph}

{\bf MODIFICATIONS}

\begin{enumerate}
\item The major difference between TS original and XDS implementations
      of the module {\bf Graph} is a windowed mode: a program, which uses the
      module, opens the \verb'Graph' window by a call to the \verb'Graph.Init'
      procedure, invokes the procedures provided by the module to draw in the
      window and set some parameters (e.g. a clipping  region) and calls to
      the \verb'Graph.Exit' procedure to close the {\bf Graph} window.
      It is possible to open only one {\bf Graph} window at a time.

      The \verb'Init' procedure has been changed to

      \verb'PROCEDURE Init(xLeft, yTop, xd, yd : LONGCARD) : BOOLEAN;'

      The position of the left upper corner of the {\bf Graph} window is passed in
      the \verb'xLeft' and \verb'yTop' parameters, horizontal and vertical sizes
      of the window are passed in the \verb'xd' and \verb'yd' parameters.
      The {\bf Graph} window is closed automatically upon program termination.
      The \verb'Exit' procedure can be used to close the {\bf Graph} window before
      the program execution is finished:

      \verb'PROCEDURE Exit;'

      The \verb'Init' procedure returns success indicator (\verb'TRUE' means succesful
      module initialization) and it is recommended to test the value in the
      program.
      
\item The procedures \verb'GraphMode' and \verb'TextMode' switch focus between
      the console window and the \verb'Graph' window and appropriately change
      their z-order.
 
\item The procedure \verb'SetGraphWindowTitle', which sets the title of the
      \verb'Graph' window, has been added:

      \verb'PROCEDURE SetGraphWindowTitle (text :ARRAY OF CHAR);'

\item The procedure \verb'GetVideoConfig' has been removed. The fields of the
      \verb'VideoConfig' record, which were typically used, are exported as variables:

      \begin{tabular}{l|l|l}
      \bf Name    &        \bf Explanation                & \bf VideoConfig field \\
      \hline
      \tt Width      & width of the Graph window work area  & \tt numxpixels            \\
      \tt Depth      & depth of the Graph window work area  & \tt numypixels            \\
      \tt NumPColors & number of colors in use              & \tt numcolors             \\
      \tt Columns    & number of text columns available     & \tt numtextcols           \\
      \tt Rows       & number of text rows available        & \tt numtextrows
      \end{tabular}

      {\bf Note:} By "number of text columns" ("text rows") a maximum number
                  of monospaced font characters which can be placed into the {\bf Graph}
                  window in one column (row) is meant.

\item The \verb'Fill' parameter has been added to the procedure \verb'Polygon':

\begin{verbatim}
PROCEDURE Polygon(n : CARDINAL;
                  px, py : ARRAY OF CARDINAL;
                  FillColor : CARDINAL;
                  Fill : BOOLEAN);
\end{verbatim}

      The variable \verb'FillState', influencing on \verb'Polygon''s
      fill mode, has been excluded.

\item The procedure \verb'Pie_a' has been added:

\begin{verbatim}
PROCEDURE Pie_a (x1, y1, x2, y2 : LONGCARD;
                 startAngle, sweepAngle : LONGREAL;
                 Color :CARDINAL; Fill : BOOLEAN);
\end{verbatim}

        A pie is defined as a part of an ellipse enclosed in the rectangle
        ( \verb'x1', \verb'y1', \verb'x2', \verb'y2' ), with a start angle
        passed in the \verb'startAngle' parameter and a sweep angle passed
        in the \verb'sweepAngle'. Angles are measured in degrees. The pie
        is drawn counterclockwise.

\item It is recommended to invoke the \verb'CancelClipRgn' procedure to disable
      clipping instead of a call to \verb'SetClipRgn' with a whole window position
      and sizes passed as parameters.

\item The procedure \verb'RawOutText' draws a text at the position (\verb'x',\verb'y')
      defined in pels rather than in text columns and rows:

\begin{verbatim}
PROCEDURE RawOutText (x, y : LONGCARD;
                      Color : LONGCARD;
                      Text :ARRAY OF CHAR);
\end{verbatim}

\item Since Win32 GDI engine keep a bitmap in a system heap,
      there is no need to allocate memory for a bitmap in a program which
      uses the module, so the procedure \verb'ImageSize' always returns 0.

      The following scheme shows the standard order of procedure calls within
      the program which uses bitmaps:

      \begin{enumerate}
      \item Invoke the \verb'GetImage' procedure to create a bitmap:

\begin{verbatim}
PROCEDURE GetImage(x1, y1, x2, y2 : LONGCARD; 
                   VAR bmpHandle : HBITMAP );
\end{verbatim}
            The handle of the created bitmap will be returned in the \verb'bmpHandle'
            parameter (the \verb'HBITMAP' type is exported from the module).

      \item To draw the previously created bitmap, call the \verb'PutImage' procedure,
            passing the bitmap handle in the \verb'bmpHandle' parameter:

\begin{verbatim}
PROCEDURE PutImage(x, y : LONGCARD; 
                   bmpHandle : HBITMAP;
                   Action : LONGCARD);
\end{verbatim}

      \item When a bitmap is no longer needed, it is recommended to delete it
            in order to release system resources by calling the 
            \verb'DelImage' procedure:

            \verb'PROCEDURE DelImage(hbmHandle :HBITMAP);'
      \end{enumerate}

\item Because of an OS/2 GPI restriction, there is no way to set an arbitrary line
      style (e.g. to draw outline circle ), nevertheless there is a number of
      the predefined line styles which can be set by means of a call to the
      \verb'SetLineStyle' procedure. Any of the following constants, exported from the
      module, can be passed to the \verb'SetLineStyle' procedure:

\begin{verbatim}
    LNSTYLE_DOT
    LNSTYLE_SHORTDASH
    LNSTYLE_DASHDOT
    LNSTYLE_DOUBLEDOT
    LNSTYLE_DASHDOUBLEDOT
    LNSTYLE_SOLID
\end{verbatim}

\item A program can define its own fill mask (the \verb'SetFillMask' procedure).
      Also the {\bf Graph} module provides the \verb'SetStdFillMask' procedure to use
      several standard fill masks. The following constants, exported from
      the module, define different fill masks :

\begin{verbatim}
    PATSYM_DENSE1
    PATSYM_DENSE2
    PATSYM_DENSE3
    ...
\end{verbatim}

\item The \verb'SetBkMix' procedure has been added to the original functionality.

          PROCEDURE SetBkMix (isOpaque : BOOLEAN);

      The \verb'_TRANSPARENT' and \verb'_OPAQUE' constants can be passed to the
      \verb'isOpague' parameter.
      The procedure sets background color mixture mode. Namely, if \verb'_OPAQUE'
      passed, bits of a fill mask, set in zero, are painted in background color
      when filling with a non-solid mask is used. If \verb'_TRANSPARENT' passed,
      screen pixels corresponding to zero bits of a fill mask, are not
      changed.

\item The \verb'RemapPalette' and \verb'RemapAllPalette' procedures are provided to set
      a logical palette. The \verb'InitStdPalette' procedure allows to set
      the palette to the initial state, in which only 16 colors, corresponding
      to the colors constants (e.g. \verb'_clrLIGHTBLUE' ), are defined.

\item The procedures \verb'TrueCircle' and \verb'TrueDisc' are defined as equal
      to \verb'Circle' and \verb'Disc' respectively, as pixels are square
      in all commonly used video modes provided by modern display adapters.
\end{enumerate}

{\bf RESTRICTIONS}

\begin{enumerate}
\item Device specific constants and routines, such as \verb'CGAWidth',
      \verb'InitCGA', etc., have been removed

\item The following procedures have been excluded:
      \verb'SelectPalette' (CGA only),
      \verb'SetVideoMode',
      \verb'SetActivePage',
      \verb'SetVisualPage',

\item It is possible to set a custom palette by means of a call to the
      \verb'RemapPalette' or \verb'RemapAllPalette' procedures. However,
      the palette colors map to the nearest colors of the current physical
      palette on your device.
\end{enumerate}

{\bf PRODUCTS}

Native XDS-x86 for Windows\\
Native XDS-x86 for Linux (with limitations)\\

%-------------------------------------------------------------------------------

\subsection{Module IO}

{\bf RESTRICTIONS}

\begin{enumerate}
\item The \verb'IO.KeyPressed()' and \verb'IO.RdKey()' procedures,
      providing direct read from keyboard, are available in console
      applications only.

\item \verb'RdReal' and \verb'RdLngReal' do not recognize exponent-only zeroes,
      such as \verb'E+0'.
\end{enumerate}

{\bf PRODUCTS}

All XDS products

%-----------------------------------------------------------------------------

\subsection{Module Lib}

{\bf MODIFICATIONS}

\begin{enumerate}
\item Although address operations (\verb'SubAddr', \verb'AddAddr' etc.) and
      other low-level procedures (such as \verb'Fill', \verb'Move') are 
      available, usage of their analogs from the module \verb'SYSTEM'
      (\verb'ADDADR', \verb'SUBADR', \verb'FILL', etc.) would improve
      performance. Please consult the User's Guide, Chapter "XDS Modula-2" 
      for more information.

\item The procedural variable \verb'RunTimeError' has been removed. In ISO
      Modula-2, exceptions should be used for error handling purposes.
      The \verb'MathError' procedure has been excluded because of
      the same reason. See also the description of changes made to the {\bf FIO} module.

\item The \verb'Terminate' procedure has been removed because the same effect
      can be obtained by means of ISO Modula-2 standard mechanisms, such as module
      finalization and the system module {\bf TERMINATION}
      (see XDS User Guide, new language's features)

\item \verb'SetDate' and \verb'SetTime' are now pure procedures.

\item The \verb'Environment' procedure definition has been changed to

\verb'    PROCEDURE Environment (N :CARDINAL;'\\
\verb'                           VAR result :ARRAY OF CHAR);'

\item The \verb'WordFill' procedure definition has been changed to

\verb'    PROCEDURE WordFill(Dest: ADDRESS;'\\
\verb'                       WordCount: CARDINAL;'\\
\verb'                       Value: SYSTEM.CARD16);'
\end{enumerate}

{\bf RESTRICTIONS}

\begin{enumerate}
\item The following platform-specific procedures have been removed:
\begin{verbatim}
      AddressOK               CpuId
      GetInProgramFlag        Execute
      NoSound                 ProtectedMode
      SelectorLimit           SetReturnCode
      SetInProgramFlag        Sound
      SysErrNo                UserBreak
      WrDosError
\end{verbatim}

\item The \verb'EnableBreakCheck' and \verb'DisableBreakCheck' procedures are
      not implemented; the \verb'CtrlC' module from the XDS run-time library
      should be used to handle \verb'Ctrl-C' and \verb'Ctrl-Break' events.
\end{enumerate}

{\bf PRODUCTS}

All XDS products

%-------------------------------------------------------------------------------

\subsection{Module MATHLIB}

{\bf RESTRICTIONS}

\begin{enumerate}
\item The following procedures are not implemented:
      \verb'LongToBcd',
      \verb'BcdToLong'

\item ISO \mt{} exception handling (see \ref{m2:ISO:features})
      is used to report floating point errors,
      therefore the procedure variable \verb'MathError' is not provided.
\end{enumerate}

{\bf PRODUCTS}

All XDS products

%-----------------------------------------------------------------------------

\subsection{Module ShtHeap}

{\bf MODIFICATIONS}

\begin{enumerate}
\item If a memory allocation failure occurs, \verb'NIL' is returned.

\item The type \verb'ShtHeap.Pointer' was replaced with the type
      \verb'SYSTEM.ADDRESS'.

\item The \verb'ShtHeap.Increase' procedure has been eliminated.

\item Heap size is not resricted by 64 Kbytes (so \verb'ShtHeap' is a 
      historical name).
\end{enumerate}

{\bf PRODUCTS}

All XDS products

%-------------------------------------------------------------------------------

\subsection{Module Str}

{\bf MODIFICATIONS}

\begin{enumerate}
\item The \verb'LowS' procedure (conversion to lower case) has been added to
      the module.
\item In TopSpeed, the procedure \verb'FixRealToStr' does not output more than 
      17 digits, whereas in XDS there is always \verb'Precision' digits after 
      decimal point.

      As a result, after a call \verb'FixRealToStr(V,5,S,Ok)' variables
      \verb'S' and \verb'Ok' have the following vaues in TS and XDS 
      respectively:

      {\tt
      \begin{tabular}{ll|ll}
      V       & SIZE(S) & S                       & OK \\
      \hline
      0.99E17 & 30      & 99000000000000000.0      & TRUE \\
              &         & 99000000000000000.00000  & TRUE \\
      0.99E18 & 30      & 990000000000000000       & TRUE \\
              &         & 990000000000000000.00000 & TRUE \\
      0.99E18 & 18      & 990000000000000000       & TRUE \\
              &         & ?                        & FALSE \\
      \end{tabular}
      }
   

\end{enumerate}

{\bf PRODUCTS}

All XDS products

%-----------------------------------------------------------------------------

\subsection{Module Window}
\index{Window}

{\bf EXTENSIONS}

\begin{enumerate}
\item An arbitrary screen size (number of rows and columns) is supported.
      The \verb'ScreenWidth' ( = 80 ) and \verb'ScreenDepth' (= 25) constants are
      remained, but in the case of non-standard screen size,
      actual values can be obtained from the \verb'CurrentScreenWidth' and
      \verb'CurrentScreenDepth' variables (the \verb'CurrentScreenWidth' variable has
      been added to the original TS Window functionality).

\item Input and output in any window, can be performed not only by means
      of the XDS TS-like module {\bf IO}, but also through ISO standard
      input/output modules, such as STextIO or SRawIO.
\end{enumerate}

{\bf RESTRICTIONS}

\begin{enumerate}
\item The \verb'CGASnow' constant has been removed.

\item It is needed to call the \verb'GetWDescriptor' procedure to obtain
      pointer to the \verb'WinDescriptor' record, since in the XDS
      implementation of the {\bf Window} module,  \verb'WinType' is an opaque type
      rather than pointer to record. The following declarations were
      added to the definition module:

\begin{verbatim}
    TYPE
      p2WDescriptor = POINTER TO WinDescriptor;

      PROCEDURE GetWDescriptor ( W :WinType ) :p2WDescriptor;
\end{verbatim}

\item The \verb'SetProcessLocks' procedures has been removed (multithread
      support implemented in such a way, that there is no need to use it).
\end{enumerate}

{\bf PRODUCTS}

Native XDS-x86 for Windows\\
Native XDS-x86 for Linux (with limitations)

\subsection{Multithreading}
\label{mt}

    The TopSpeed module \verb'Process' implements its own scheduler which does
    not make use of the underlying operating system multithreading capabilities,
    in order to enable the development of multithread programs portable between
    DOS and OS/2. In case of \xds{}, the ISO \verb'Processes' module should be used.
    Under Windows, its functionality is mapped onto
    the corresponding thread control API.

    {\bf Note:} The {\bf MULTITHREAD} option should be set on to enable the
    functionality described above. Otherwise, the module \verb'Processes' will
    work in coroutines mode.

    An example of the \verb'Processes' module usage is the \verb'WinDemoM'
    sample which was ported from TopSpeed \mt{}.

    The \verb'Processes' module does not provide direct way to start or stop the
    process scheduler, which is possible in TopSpeed \verb'Process' module, but this
    problem can be easily solved by writing some extra code (see \verb'WinDemoM.mod').

    The ISO standard module \verb'Semaphores' can be used to utilize the underlying
    operating system semaphore API.


