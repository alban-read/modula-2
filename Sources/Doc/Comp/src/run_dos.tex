\section{Running a program}\label{start:run}
\index{running a program}

\ifgencode        %----------------------
  After compilation of all modules composing your project you have
  to link the program. It can be done using a linker from your
  C compiler package. In this section examples of using linkers
  from Symantec (v6.x, for DOSX extender) and Watcom (v10.x,
  DOS4/GW extender) are provided (See also Chapter \ref{ccomp}).

  \paragraph{To link a program (Symantec):}
\begin{verbatim}
  sc -mx hello.obj \xds\lib\xds_sc.lib
\end{verbatim}

  \paragraph{To link a program (Watcom):}
\begin{verbatim}
  wcl386 /"op c" hello.obj \xds\lib\xds_wc.lib
\end{verbatim}
  After that, one can invoke the program:
  \begin{flushleft} \tt
        hello
  \end{flushleft}

  It may be necessary to specify additional options to put the debug
  information into {\tt hello.exe} file, or to specify the stack size for
  DOS4GW. For Watcom compiler the option "op c" (case-sensitive) should
  be set.

  To simplify linking, \XDS{} allows one to specify a linker command line
  in the \XDS{} environment, using the {\bf LINK} equation.
  A command line
  specified by the equation will be executed after successful
  compilation of the whole project. Usually, the equation is specified
  in the project file (See \ref{xc:project}).

\paragraph{Example of the project file {\tt hello.prj} (Symantec)}
\begin{verbatim}
-link="sc -mx hello.obj \\xds\\lib\\xds_sc.lib"
!module hello.mod
\end{verbatim}

\paragraph{Example of the project file {\tt hello.prj} (Watcom)}
\begin{verbatim}
-link='wcl386 /"op c" hello.obj \\xds\\lib\\xds_wc.lib'
!module hello.mod
\end{verbatim}
The project file contains the {\bf LINK} equation and
a name of a program module.
The following invokation
\begin{flushleft} \tt
        \xc{} hello.prj =project
\end{flushleft}
will compile modules constituting the project (if required) and
then execute the linker command line specified.

As can be seen, a linker command line consists of a name of a linker
(or C compiler), options, a list of object files and required libraries.
\XDS{} allows a linker command to be specified in a generic form, so
a list of object files constituting your program will be generated
automatically. It can be done using the so called {\em iterator}
(See \ref{xc:template}). An iterator of the form
\begin{verbatim}
{ " %s",imp+main+oberon#objext; }
\end{verbatim}
will insert the names of all implementation modules,
all oberon modules and the name of a program module in the command line.
Thus, the value of a link equation shall look like
\begin{verbatim}
<linker> <options> <iterator> <library>
\end{verbatim}

\paragraph{For Symantec:}
\begin{flushleft}
\begin{tabular}{ll}
linker  & \verb|sc|                            \\
options & \verb|-mx|                           \\
library & \verb|c:\xds\lib\xds_sc.lib|         \\
\end{tabular}
\end{flushleft}

\paragraph{For Watcom:}
\begin{flushleft}
\begin{tabular}{ll}
linker  & \verb|wl386|                            \\
options & \verb|/"deb codeview op c op st=128k"|  \\
library & \verb|c:\xds\lib\xds_wc.lib|            \\
\end{tabular}
\end{flushleft}
Here the stack size (128k) and the format of debug
information (CodeView) are specfied.

If your project contains more than one module, the linker command
line can be too long. Usually under MS-DOS a command line cannot
exceed 128 symbols.
It can be more convenient to link such projects using
the {\em make} utility. The \XDS{} distribution contains
template files ({\tt symantec.tem} and {\tt watcom.tem}),
that are used to generate a makefile.

\paragraph{Example of the project file (Symantec)}
\begin{verbatim}
-LINK='make -f %s',project#'mkf';
-makefile+
-template=\xds\bin\symantec.tem
!module hello.mod
\end{verbatim}

\paragraph{Example of the project file (Symantec)}
\begin{verbatim}
-LINK='wmake -f %s',project#'mkf';
-makefile+
-template=\xds\bin\watcom.tem
!module hello.mod
\end{verbatim}

For more details see \ref{xc:project} and \ref{xc:template}.

\fi % end ifgencode ----------------------
